{\bf Subsets}
\hrule
\vskip 6pt

$\bullet$ A set $A$ is called a {\bf subset} of a set $B$ if every element of $A$ also belongs to $B$. If $A$ is a subset of $B$, then we write $$A \subseteq B$$

$\bullet$ We are concerned with subsets of some specified set $U$, called the {\bf universal set}.
\vskip 1pc

$\bullet$ For $a,b \in R$ and $a<b$, the {\bf open interval} $a,b$ is the set $$(a,b)= {\lbrace x\in R: a<x<b \rbrace}$$
\vskip 1pc

$\bullet$ For $a,b \in R$ and $a\leq b$, the {\bf closed interval} $a,b$ is the set $$\lbrack a,b\rbrack= {\lbrace x\in R: a\leq x\leq b \rbrace}$$
\vskip 1pc

$\bullet$ For $a<b$, we have $(a,b)\subseteq [a,b]$. For $a,b \in R$ and $a<b$, the {\bf half-open} or {\bf half-closed intervals} $[a,b)$ and $(a,b]$ are defined as $$[a,b)= \lbrace x\in R: a\leq x < b\rbrace \quad\hbox{and}\quad (a,b]= \lbrace x\in R: a< x\leq b\rbrace $$
\vskip 1pc

$\bullet$ For $a\in R$, the infinite intervals $(-\infty,a),(-\infty,a],(a,\infty), [a,\infty)$
\vskip 1pc

$\bullet$ Two sets $A$ and $B$ are {\bf equal}, indicated by writing $A=B$, if they have exactly the same elements.
\vskip 1pc

$\bullet$ A set $A$ is a {\bf proper subset} of a set $B$ if $A \subseteq B$ but $A\neq B$
\vskip 1pc

$\bullet$ It is convient to represent sets as {\bf Venn diagrams}.
\vskip 1pc

$\bullet$ The set consisting of all subsets of a given set is called the {\bf power set} of $A$ and is denoted $\varphi(A)$

\vfill\eject
