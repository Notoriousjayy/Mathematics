{\bf Partitions of Sets}
\hrule
\vskip 6pt

$\bullet$ A collection $S$ of subsets of a set $A$ is called {\bf pairwise disjoint} if every two distinct subsets that belong to $S$ are disjoint.
\vskip 1pc

$\bullet$ A {\bf partition} of $A$ can be defined as a collection $S$ of non-empty subsets of $A$ such that every element of $A$ belongs to exactly one subset in $S$
\vskip 1pc

$\bullet$ There are three ways that a collection of $S$ of subsets of a non-empty set $A$ is defined to be a partition of $A$
\vskip 1mm

{\bf Definition 1} The collection $S$ consits of pairwise disjoint non-empty subsets of $A$ and every element of $A$ belongs to a subset in $S$.
\vskip 1mm

{\bf Definition 2} The collection $S$ consists of non-empty subsets of $A$ and every element of $A$ belongs to exactly one subset in $S$
\vskip 1mm

{\bf Definition 3} The collection $S$ consists of subsets of $A$ satisfying the three properties:
\vskip 1mm

\centerline{$(1)$ every subset in $S$ is non-empty and,}
\vskip 1mm

\centerline{$(2)$ every two subsets of $A$ are equal or disjoint and}
\vskip 1mm

\centerline{$(3)$ the union of all subsets in $S$ is $A$.}

\vfill\eject
