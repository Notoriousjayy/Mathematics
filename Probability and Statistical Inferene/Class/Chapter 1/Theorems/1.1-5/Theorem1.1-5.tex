\nopagenumbers
{\bf Theorem: 1.1-5}
\vskip 1mm
\hrule

\vskip 6pt
If $A$ and $B$ are any two events, then

$$P(A\cup B)=P(A)+P(B)-P(A\cap B)$$

\vskip 10pt
{\bf Proof}

\vskip 6pt
The event $A\cup B$ can be represented as a union of mutually exclusive events, namey

$$A\cup B= A\cup (A' \cap B)$$

Hence, by property (c) of {\bf probability}

$$P(A\cup B)=P(A)+P(A'\cap B)$$

However,

$$B=(A\cap B)\cup (A'\cap B)$$

which is a union of mutually exclusive events. Thus

$$P(B)=P(A\cap B)+P(A'\cap B)$$

and

$$P(A'\cap B)=P(B)-P(A\cap B)$$

If the right side of this equation is substituted into $P(A\cup B)=P(A)+P(A'\cap B)$, we obtain

$$P(A\cup B)=P(A)+P(B)-P(A\cap B)$$

which is the desired result.


\vfill\eject
\bye
