{\bf Vectors in Space}
\vskip 1mm
\hrule

\vskip 6pt
{\bf What Do We Mean by $R^n$?}

\vskip 6pt
$R^n$ for a given integer $n$ stands for the collection of all possible strings

$$(a_1,a_2,a_3,\cdots,a_n)$$

Such a string is called $n$-tuple of nummbes.

\vskip 1mm
When $n=1$ all we have is a real number.

\filbreak
\vskip 1cm
{\bf Addition, Subtraction, and Scalar Multiplication in $R^n$}

\vskip 6pt
We suppose that $n$ is a given positive integer and that $A=(a_1,a_2,a_3,\cdots,a_n)$ and $B=(b_1,b_2,b_3,\cdots,b_n)$ are points in $R^n$.

\vskip 1mm
We define

$$A+B=(a_1,a_2,a_3,\cdots,a_n)+(b_1,b_2,b_3,\cdots,b_n)=(a_1+b_1,a_2+b_2,\cdots a_n+b_n)$$

and

$$A-B=(a_1,a_2,a_3,\cdots,a_n)-(b_1,b_2,b_3,\cdots,b_n)=(a_1-b_1,a_2-b_2,\cdots a_n-b_n)$$

We suppose that $n$ is a given integer, that $A=(a_1,a_2,a_3,\cdots,a_n)$, and that $t$ is any given number. The product $tA$ is defined by the equation

$$tA=t(a_1,a_2,a_3,\cdots,a_n)=(ta_1,ta_2,ta_3,\cdots,ta_n)$$

\filbreak
\vskip 1cm
{\bf The Norm $||A||$ of a point $A$ in $R^n$}

If $A=(a_1,a_2,a_3,\cdots,a_n)$ then we define

$$||A||=\sqrt{a{}_1^2,a{}_2^2,a{}_3^2,\cdots,a{}_n^2}$$

\filbreak
\vskip 1cm
{\bf Special Case: The Norm $||A||$ of a point $A$ in $R^n$}

\vskip 1mm
The norm $||(x,y)||$ of a point $(x,y)$ in $R^2$ is $\sqrt{x^2+y^2}$

\filbreak
\vskip 1cm
{\bf The Norm of a Point $P$ in $R^3$}

\vskip 1mm
In $R^3$ the number $||P||$ is the distance from $P$ to $O$.

\vskip 1mm
Applying the theorem of Pythagoras to $\triangle OQP$ we see that

$$\eqalign{(OP)^2&=(OQ)^2+(QP)^2\cr
		&= (\sqrt{x^2+y^2})^2+z^2\cr
		&=x^2+y^2+z^2\cr
		OP&=\sqrt{x^2+y^2+z^2}=||P||\cr}$$

The norm of a point is its distance to the point $O=(0,0,0)$.

\filbreak
\
\vskip 1cm
{\bf Some Properties of the Arithmetic in $R^n$}

\vskip 1mm
Assume that $A=(a_1,a_2,a_3,\cdots,a_n)$ and $B=(b_1,b_2,b_3,\cdots,b_n)$ and $C=(c_1,c_2,c_3,\cdots,c_n)$ and that $s$ and $t$ are numbers.

\vskip 1cm
{\bf The Commutative Law $A+B=B+A$}

\vskip 1mm
{\bf Proof}

$$\eqalign{A+B&=(a_1+b_1,a_2+b_2,\cdots,a_n+b_2)\cr
		&=(b_1+a_1,b_2+a_2,\cdots,b_n+a_n)\cr
		&=B+A\cr}$$

\filbreak
\vskip 1cm
{\bf The Associative Law $(A+B)+C=A+(B+C)$}

\vskip 1mm
{\bf Proof}

\vskip 1mm
The left side is

$$\eqalign{a_1+b_1,a_2+b_2,\cdots,a_n+b_n+(c_1,c_2,c_3,\cdots,c_n)&=((a_1+b_1)+c_1,(a_2+b_2)+c_2,\cdots,(a_n+b_n)+c_n)\cr
									&=(a_1+b_1+c_1,a_2+b_2+c_2,\cdots,a_n+b_n+c_2)\cr}$$

\filbreak
\vskip 1cm
{\bf The Point $O$}

\vskip 1mm
This is the point $(0,0,0)$

\vskip 1mm
The Equation $A+O=A$

\vskip 1mm
The Equation $A-A=0$

\vskip 1mm
The Equation $1A=A$

\vskip 1mm
The Equation $0A=O$

\vskip 1mm
The Equation $(s+t)A=sA+tA$

\vskip 1mm
The Equation $t(A+B)=tA+tB$

\vskip 1mm
The Equation $||tA||=|t|\,||A||$

\vskip 1mm
{\bf Proof}

\vskip 1mm
I want to point out that, if $t$ is any real number, then $\sqrt{t^2}=|t|$

$$\eqalign{||tA||&=||t(a_1,a_2,a_3,\cdots,a_n)||\cr
		&=(ta_1,ta_2,ta_3,\cdots,ta_n)||\cr
		&=\sqrt{(ta_1)^2+(ta_2)^2+(ta_3)^2+\cdots+(ta_n)^2}\cr
		&=\sqrt{t^2(a{}_1^2+a{}_2^2+\cdots a{}_n^2)}\cr
		&=|t|=\sqrt{a{}_1^2+a{}_2^2+\cdots a{}_n^2}\cr
		&=|t|\,||A||\cr}$$

\filbreak
\vskip 1cm
{\bf The Standard Basis in $R^n$}

\vskip 1mm
The {\bf standard basis} in $R^3$ is the set $\{(1,0,0),(0,1,0),(0,0,1)\}$ and we can extend this to $R^n$

\vskip 1mm
We write $(1,0,0)=I$ and $(0,1,0)=J$ and $(0,0,1)=K$

\vskip 1mm
In $R^n$ we define

$$\eqalign{I_1&=(1,0,0,\cdots,0)\cr
	I_2&=(0,1,0,\cdots,0)\cr
	I_3&=(0,0,1,0,\cdots,0)\cr
	\vdots\cr
	I_n&=(0,0,0,\cdots,1)\cr}$$

and the set $\{I_1,I_2,I_3,\cdots,I_n\}$ is called the {\bf standad basis} in $R^n$.

\filbreak
\vskip 1cm
{\bf Linear Combinations}

\vskip 1mm
A {\bf linear combination} of points $A$ and $B$ and $C$ is any point of the form

$$rA+sB+tC$$

where $r$ and $s$ and $t$ can be any numbers. In the same way, we can have a linear combination of many points.

\vskip 1mm
We can extend this idea to larger sets. For example, since

$$(3,-2,5,0,7)=3(1,0,0,0,0)-2(0,1,0,0,0)+(0,0,1,0,0)+0(0,0,0,1,0)+7(0,0,0,0,1)$$

we see that $(3,-2,5,0,7)$ is a linear combination of the set $\{I_1,I_2,I_3,I_4,I_5\}$

\filbreak
\vskip 1cm
{\bf Linear Independence}

\vskip 1mm
A set of points in $R^n$ is said to be {\bf linearly independent} if it is impossible to make linear combination of these points equal to $O$ unless all the coeffiencients are zero.

\vskip 1mm
For example, since

$$a(1,0,0,0)+b(0,1,0,0)+c(0,0,1,0)+d(0,0,0,1)=(a,b,c,d)$$

the only way we can make a linear combination

$$a(1,0,0,0)+b(0,1,0,0)+c(0,0,1,0)+d(0,0,0,1)$$

equal to $O$ is by making $a=b=c=d$

\filbreak
\vskip 1cm
{\bf An Important Principle Studied in a First Course in Linear Algebra}

\vskip 1mm
If a set $\{A,B,C\}$ of points in $R^3$ is linearly independent, then {\it every} point in $R^3$ can be expressed as a linear combination of $A,B,$ and $C$.

\vskip 1mm
In other words, if $W$ is any given point in $R^3$, then the equation

$$xA+yB+zC=W$$

can be solved for $x,y,$ and $z$

\vskip 1mm
So, for example, if $A=(a_1,a_2,a_3)$ and $B=(b_1,b_2,b_3)$ and $C=(c_1,c_2,c_3)$ and $W=(w_1,w_2,w_3)$, then the equation says that

$$x(a_1,a_2,a_3)+y(b_1,b_2,b_3)+z(c_1,c_2,c_3)=(w_1,w_2,w_3)$$

and this says that

$$\eqalign{a_1x+b_1y+c_1z&=w_1\cr
		a_2x+b_2y+c_2z&=w_2\cr
		a_3x+b_3y+c_3z&=w_3\cr}$$

and we are saying that this system of three equations in the three unknowns $x,y,$ and $z$ can be solved. We can make a similiar statement for $R^n$ for each $n$.

\filbreak
\vskip 1cm
{\bf Geometric Interpretation of the Arithmetic in $R^2$ and $R^3$}

\vskip 1cm
{\bf Line Segments with the Same Length and Direction}

\vskip 1mm
We suppose that we have four points

$$\eqalign{A&=(x_1,y_1,z_1)\cr
		B&=(x_2,y_2,z_2)\cr
		C&=(x_3,y_3,z_3)\cr
		D&=(x_4,y_4,z_4)\cr}$$

in $R^3$ and we look at the two line segments $AB$ and $CD$.

\vskip 1mm
The conditions for these two line segments to have the same length and also point in the same direction is that the equations

$$\eqalign{x_2-x_1&=x_4-x_3\cr
	y_2-y_1&=7y_4-y_3\cr
	z_2-z_1&=z_4-z_3\cr}$$

hold. Thus the two line segments $AB$ and $CD$ will have the same length and direction if and only if

$$\eqalign{(x_2-x_1,y_2-y_1,z_2-z_1)&=(x_4-x_3,y_4-y_3,z_4-z_3)\cr
		B-A&=D-C\cr}$$

because this says that 

$$(x_2-x_1,y_2-y_1,z_2-z_1)=(x_4-x_3,y_4-y_3,z_4-z_3)$$

We can, of course, make the same observation about points in $R^2$.

\filbreak
\vskip 1cm
{\bf The Notation $\vec{AB}$}

\vskip 1mm
Given any two points $A$ and $B$ in $R^n$ we define

$$\vec{AB}=B-A$$

Given points $A$ and $B$ and $C$ and $D$ in space, the condition

$$\vec{AB}=\vec{CD}$$

says that

$$B-A=D-C$$

and this is the condition $AB$ and $CD$ are line segments with the same length and the same direction.

\filbreak
\vskip 1cm
{\bf We need a name for $\vec{AB}$}

\vskip 1mm
The symbol $\vec{AB}$ is given the name $AB$

\vskip 1mm
Note that, if $A$ and $B$ are points in space then the symbol $AB$ stands for the line segment between $A$ and $B$ and, sometimes, we use the symbol $AB$ for the {\it length} of the line segment $AB$.

\vskip 1mm
On the other hand, vector $AB$, which is written as $\vec{AB}$, means a {\bf point} in space. In fact, $\vec{AB}$ is the point $B-A$.

\vskip 1mm
So, for example, if $A=(2,4,-3)$ and $B=(5,1,-8)$, then

$$\eqalign{\vec{AB}&=B-1\cr
	&=(5,1,-8)-(2,4,-3)\cr
	&=(3,-3,-5)\cr}$$

\filbreak
\vskip 1cm
{\bf Using the Norm to Find the Length of a Line Segment in $R^3$}

\vskip 1mm
If $A=(x_1,y_1,z_1)$ and $B=(x_2,y_2,z_2)$ are any points in $R^3$, then the distance formula tells us that the distance from $A$ to $B$ is

$$\eqalign{\sqrt{(x_2-x_1)^2+(y_2-y_1)^2+(z_2-z_1)^2}&=||(x_2-x_1,y_2-y_1,z_2-z_1)||\cr
							&=||B-A||\cr
							&=||\vec{AB}||\cr}$$

Another way of looking at this topic is to define $P=B-A$. From the equation

$$P-O=P=B-A$$

 we see that the line segments $OP$ and $AB$ have the same length (and direction) and so

$$dis(AB)=dist(OP)=||P||=||B-A||=\sqrt{(x_2-x_1)^2+(y_2-y_1)^2+(z_2-z_1)^2}$$

\filbreak
\vskip 1cm
{\bf Line Segments with the Same Length}

\vskip 1mm
Two line segments $AB$ and $CD$ have the same length if and only if $||\vec{AB}||=||\vec{CD}||$

$$||B-A||=||D-C||$$

\filbreak
\vskip 1cm
{\bf Two Line Segments with the Same Direction}

\vskip 1cm
{\bf We begin by looking at a special case}

\vskip 1mm
Suppose that $AB$ and $CD$ have the same direction but that $AB$ is twice as long as $CD$ and that

$$\eqalign{A&=(x_1,y_1,z_1)\cr
	B&=(x_2,y_2,z_2)\cr
	C&=(x_3,y_3,z_3)\cr
	D&=(x_4,y_4,z_4)\cr}$$

In this case

$$\eqalign{x_2-x_1&=2(x_4-x_3)\cr
	y_2-y_1&=2(y_4-y_3)\cr
	z_2-z_1&=2(z_4-z_3)\cr}$$

Un this case we see that

$$B-A=2(D-C)$$

and this says that

$$\vec{AB}=2\vec{CD}$$

\vskip 1mm
{\bf Now we extend the idea}

\vskip 1mm
Suppose that $t$ is a positive number. The condition

$$B-A=t(D-C)$$

which is 

$$\vec{AB}=t\vec{CD}$$

says that the line segments $AB$ and $CD$ have the same direction but that $AB$ is $t$ times as long as $CD$

\filbreak
\vskip 1cm
{\bf Line Segments with Opposite Directions}

\vskip 1mm
1. Given points $A$ and $B$ we can see that

$$\vec{BA}=A-B=-(B-A)=-\vec{AB}$$

and we can see, the directions of the line segments $AB$ and $BA$ are opposite to one another.

\vskip 1mm
2. Now suppose that $A,B,C,$ and $D$ are points in space and that $\vec{AB}=-\vec{CD}$. This tells us that

$$\vec{AB}=B-A=-(D-C)=C-D=\vec{DC}$$

and so the line segments $AB$ and $CD$ have the same length but opposite directions.

\vskip 1mm
3. Now suppose that $A,B,C,$ and $D$ are point in space ad that $t$ is a negative number and that 

$$\vec{AB}=t(\vec{CD})$$

This gives us

$$\eqalign{\vec{AB}&=t(D-C)\cr
		&=-t(C_D)\cr
		&=(-t)(\vec{DC}\cr}$$

The number $it$ is positive. The line segment $AB$ is $-t$ times as long as the line segment $CD$ and the directions of the line segments $AB$ and $CD$ are opposite to one another.

\filbreak
\vskip 1cm
{\bf The Inner Product (Dot Product)}

\vskip 1mm
{\bf Definition of the Inner Product (Good in $R^n$ for any positive integer $n$)}

\vskip 1mm
Given two points $A=(a_1,a_2,a_3,\cdots,a_n)$ and $B=(b_1,b_2,b_3,\cdots,b_n)$ in $R^n$, we define

$$A\cdot B=a_1b_1+a_2b_2+a_3b_3+\cdots a_nb_n$$

Note that $A\cdot B$ is a number.

\vskip 1mm
{\bf When We Are in $R^3$}

\vskip 1mm
If $A=(x_1,y_1,z_1)$ and $B=(x_2,y_2,z_2)$ then

$$A\cdot B=x_1x_2+y_1y_2+z_1z_2$$

\filbreak
\vskip 1cm
{\bf Simple Facts About the Dot Product}

\vskip 1mm
We suppose that $A=(a_1,a_2,a_3,\cdots,a_n)$ and $B=(b_1,b_2,b_3,\cdots,b_n)$ and $C=(c_1,c_2,c_3,\cdots,c_n)$ and that $s$ and $t$ are numbers

\vskip 1mm
The Equation $A\cdot A=||A||^2$

\vskip 1mm
The reason for this is

$$\eqalign{A\cdot A&=(a_1,a_2,a_3,\cdots,a_n)\cdot(a_1,a_2,a_3,\cdots,a_n)\cr
		&=a{}_1^2+a{}_2^2+a{}_3^2+\cdots a{}_n^2\cr
		&=||A||^2\cr}$$

\filbreak
\vskip 1cm
{\bf The Commutative Law $A\cdot B=B\cdot A$}

$$A\cdot(B+C)=A\cdot B+A\cdot C$$

\filbreak
\vskip 1cm
{\bf The Distributive Law $A\cdot(B+C)=A\cdot B+A\cdot C$}

\filbreak
\vskip 1cm
The equation $t(A\cdot B)=(tA)\cdot B= A\cdot (tB)$

$$\eqalign{t(A\cdot B)&=t(a_1b_1+a_2b_2+\cdots+a_nb_n)\cr
		&=ta_1b_1+ta_2b_2+\cdots+ta_nb_n\cr
		&=a_1tb_1+a_2tb_2+\cdots+a_ntb_n\cr
		&=A\cdot(tB)\cr}$$

\filbreak
\vskip 1cm
{\bf The Cauchy-Schwarz Inequality $|A\cdot B|\leq||A||||B||$}

\vskip 1mm
If $\angle AOB=\theta$, then

$$A\cdot B=(OA)(OB)\cos(\theta)=||A||\;||B||\cos(\theta)$$

and so

$$\eqalign{|A\cdot B|&=||A||\;||B|||\cos(\theta)|\cr
			&\leq ||A||\;||B||1\cr}$$

\filbreak
\vskip 1cm
{\bf The Minkoski Inequality $||A+B||\leq||A||+||B||$}

$$||A+B||=||C|$$

which is the distance from $O$ to $C$ and this can't be more than the distance from $O$ to $A$ plus the distance from $O$ to $B$.

\filbreak
\vskip 1cm
{\bf The Triangle Inequality $||A-C||\leq||A-B||+||B-C||$}

\vskip 1mm
Given the points $A$ and $B$ and $C$ we have

$$||A-C||\leq||A-B||+||B-C||$$

\filbreak
\vskip 1cm
{\bf Review of the Law of Cosines}

\vskip 1mm
Explain why

$$\eqalign{a^2&=b^2+c^2-2bc\cos\angle A\cr
	b^2&=a^2+c^2-2ac\cos\angle B\cr
	c^2&=a^2+b^2-2ab\cos\angle C\cr}$$

We suppose that $\triangle ABC$ is any triangle and we move $\triangle ABC$ into a coordinate system as follows.

\vskip 1mm
1. we slide $\triangle ABC$ until $A$ is at $O$.

\vskip 1mm
2. we roate $\triangle ABC$ until $B$ is on the right side of the $x$-axis.

\vskip 1mm
3. If necessary, we flip $\triangle ABC$ to make $C$ sit above the $x$-axis.

\vskip 1mm
$$\eqalign{A&=(0,0)\cr
		B&=(c,0)\cr}$$

The point $C$ is a bit harder.

\vskip 1mm
The angle $\angle A$ is drawn in standard position. The angle winds to the terminal line $OC=AC$ whose length is $b$ and so, if we write $C$ as $(x,y)$ just for the moment, then we have

$$\eqalign{\cos\angle A&={x\over n}\cr
	\sin\angle A&={y\over b}\cr}$$

and so

$$\eqalign{C&=(b\cos\angle A, b\sin\angle A)\cr
		A&=(0,0)\cr
		B&=(c,0)\cr
		C&=(b\cos\angle A,b\sin\angle A)\cr}$$

From the distance formula we can see that

$$(BC)^2=(b\cos\angle A-c)^2+(b\sin\angle A-0)^2$$

and so

$$\eqalign{a^2&=b^2\cos^2\angle A-2bc\cos\angle A+c^2+b^2\sin^2\angle A\cr
		a^2&=b^2\cos^2\angle A+b^2\sin^2\angle -2bc\cos\angle A+c^2\cr
		a^2&=b^2(\cos^2\angle A+\sin^2\angle A)-2bc\cos\angle A+c^2\cr
		a^2&=b^2(1)-2bc\cos\angle A+c^2\cr}$$

\filbreak
\vskip 1cm
{\bf A Geometric Interpretation of the Inner Product in $R^2$ and $R^3$}

\vskip 1mm
If $A=(x_1,y_1,z_1)$ and $B=(x_2,y_2,z_2)$ are points in $R^3$ (or $R^2$) that are unequal to the origin $O$ and $\theta=\angle AOB$. Then

$$A\cdot B=||A||\;||B||\cos(\theta)$$

\vskip 1mm
{\bf Proof}

\vskip 1mm
We apply the law of cosines to $\triangle AOB$ and get

$$\eqalign{(AB)^2&=(OA)^2+(OB)^2-2(OA)(OB)\cos(\theta)\cr
	(x_2-x_1)^2+(y_2-y_1)^2+(z_2-z_1)^2&=x{}_1^2+y{}_1^2+z{}_1^2+x{}_2^2+y{}_2^2+z{}_2^2-2||A||\;||B||\cos(\theta)\cr
	x{}_1^2-2x_1x_2+x{}_2^2+y{}_1^2-2y_1y_2+y{}_2^2+z{}_1^2-2z_1z_2+z{}_2^2&=x{}_1^2+y{}_1^2+z{}_1^2+x{}_2^2+y{}_2^2+z{}_2^2-2||A||\;||B||\cos(\theta)\cr
	-2x_1x_2-2y_1y_2-2z_1z_2&=-2||A||\;||B||\cos(\theta)\cr
	x_1x_2+y_1y_2+z_1z_2&=||A||\;||B||\cos(\theta)\cr
	A\cdot B &=||A||\;||B||\cos(\theta)\cr}$$

\filbreak
\vskip 1cm
{\bf Points Orthogonal to One Another}

\vskip 1mm
We say that $A$ and $B$ are {\bf orthogonal} to each other when $A\cdot B=0$

\filbreak
\vskip 1cm
{\bf Orthogonality in $R^2$ or $R^3$}

Saying that $A\cdot B=0$ is saying that

$$||A||\;||B||\cos\angle AOB=0$$

and this gives us

$$\cos\angle AOB=0$$

and this holds when $\angle AOB=90^\circ$ and this says that $OA\perp OB$

\filbreak
\vskip 1cm
{\bf A condition for Two Line Segments to be Perpendicular to One Another}

\vskip 1mm
We define $P=B-A=\vec{AB}$ and $Q=D-C=\vec{CD}$.

\vskip 1mm
The line segments $AB$ and $OP$ have length and the same direction.

\vskip 1mm
The line segments $CD$ and $OQ$ have the same length and the samwe direction.

\vskip 1mm
The condition for $CD$ to be perpendicular to $AB$ is that $OP\perp OQ$ and this says that $P\cdot Q=0$ and this says that

$$(\vec{AB}\cdot(\vec{CD})=0$$

\filbreak
\vskip 1cm
{\bf The Points $A+B$ and $A-B$ are Orthogonal to Each other if and only if $||A||=||B||$}

\vskip 1mm
To say that $A+B$ iw orthogonal to $A-B$ is to say that

$$(A+B)\cdot(A-B)=0$$

and this says that

$$\eqalign{A\cdot A+B\cdot A-A\cdot B-B\cdot B&=0\cr
		A\cdot A+A\cdot B-A\cdot B-B\cdot B&=0\cr}$$

and this says that

$$||A||^2-||B||^2=0$$

and says that

$$||A||=||B||$$

\filbreak
\vskip 1cm
{\bf The Cross Product}

\vskip 1mm
{\bf The Definition of the $\times$ Product in $R^3$}

\vskip 1mm
Given $A=(x_1,y_1,z_1)$ and $B=(x_2,y_2,z_2)$ we define

$$\eqalign{A\times B&=(x_1,y_1,z_1)\times(x_2,y_2,z_2)\cr
		&=(y_1z_2-z_1y_2,z_1x_2-x_1z_2,x_1y_2-y_1x_2)\cr}$$

Another way to get this is to work out

$$det\left[\matrix{I&J&K\cr
		x_1&y_1&z_1\cr
		x_2&y_2&z_2\cr}\right]\eqalign{&=I(y_1z_2-y_2z_1)+J(x_2z_1-x_1z_2)+K(x_1y_2-x_2y_1)\cr
	&=(y_1z_2-y_2z_1,x_2z_1-x_1z_2.x_1z_2,x_1y_2-x_2y_1)\cr}$$

\filbreak
\vskip 1cm
{\bf The Product $A\times A$}

\vskip 1mm
Given $A=(x,y,z)$ we have

$$\eqalign{A\times A&=(x,y,z)\times(x,y,z)\cr
	&=(yz-zy,zx-xz,xy-yx)\cr
	&=(0,0,0)\cr
	&=O\cr}$$

\filbreak
\vskip 1cm
{\bf Some Facts About the $\times$ Product}

\vskip 1mm
{\bf The Equation $B\times A=-A\times B$}

\vskip 1mm
We assume that $A=(x_1,y_1,z_1)$ and $B=(x_2,y_2,z_2)$

$$\eqalign{A\times B&=(x_1,y_1,z_1)\times(x_2,y_2,z_2)\cr
		&=(y_1z_2-z_1y_2,z_1x_2-x_1z_2,x_1y_2-y_1x_2)\cr}$$

On the other hand

$$\eqalign{B\times A=(x_2,y_2,z_2)\times(x_1,y_1,z_1)\cr
		&=(y_2z_1-z_2y_1,z_2x_1-x_2z_1,x_2y_1-y_2x_1)\cr}$$

$$B\times A=-A\times B$$

\vskip 1mm
{\bf The Equation $A\times(B+C)=A\times B+A\times C$}

\vskip 1mm
We assume that $A=(x_1,y_1,z_1)$ and $B=(x_2,y_2,z_2)$ and $C=(x_3,y_3,z_3)$ and we see that

$$\eqalign{A\times (B+C)&=(x_1,y_1,z_1)\times (x_2+x_3,y_2+y_3,z_2+z_3)\cr
				&=\Bigl(y_1(z_2+z_3)-z_1(y_2+y_3),z_1(x_2+x_3)-x_1(z_2+z_3),x_1(y_2+y_3)-y_1(x_2+x_3)\Bigr)\cr}$$

\filbreak
\vskip 1cm
{\bf The Equation $(tA)\times B=t(A\times B)$}

\vskip 1mm
{\bf Failure of the Law $A\times (B\times C)=(A\times B)\times C$}

\filbreak
\vskip 1cm
{\bf The Norm of $A\times B$}

\vskip 1mm
Given $A$ and $B$ we have the following equation

$$||A\times B||^2=||A||^2||B||^2-(A\cdot B)^2$$

\vskip 1mm
If $A=(x_1,y_1,z_1)$ and $B=(x_2,y_2,z_2)$ then

$$\eqalign{||A\times B||^2&=||y_1z_2-z_1y_2,z_1x_2-x_1z_2,x_1y_2-y_1x_2||^2\cr
		&=(y_1z_2-z_1y_2)^2+(z_1x_2-x_1z_2)^2+(x_1y_2-y_1x_2)^2\cr}$$

\vskip 1mm
We can make an important conclusion

\vskip 1mm
If $\theta=\angle AOB$ then we know that

$$A\cdot B=||A||\;||B||\cos(\theta)$$

Now we use the equation

$$||A\times B||^2=||A||^2||B||^2-(A\cdot B)^2$$

and we can see that

$$\eqalign{||A\times B||^2&=||A||^2||B||^2-||A||^2||B||^2\cos^2(\theta)\cr
		&=||A||^2||B||^2(1-\cos^2(\theta)\cr
		&=||A||^2||B||^2\sin^2(\theta)\cr}$$

and so

$$||A\times B||=||A||\;||B|||\sin(\theta)|$$

Our understanding of $\theta$ is that $0\leq\theta\leq180^\circ$ and so $\sin(\theta)$ can't be negative and so we can drop the absolute value sign

$$||A\times B||=||A||\;||B||\sin(\theta)$$

\filbreak
\vskip 1cm
{\bf The Direction of $A\times B$}

\vskip 1mm
We write $P=A\times B$ and I want to explain that $OP$ is perpendicular both to $OA$ and to $OB$ and so $OP$ is perpendicular to $\triangle OAB$

\filbreak
\vskip 1cm
{\bf The Vector Triple Product}

\vskip 1mm
Given $A$ and $B$ and $C$

$$A\times(B\times C)=(A\times C)B-(A\cdot B)C$$

\filbreak
\vskip 1cm
{\bf Lines and Planes in $R^3$}

\vskip 1mm
{\bf Equation of a Plane}

\vskip 1mm
We can specify a plane in $R^3$ by giving a point $A=(x_1,y_1,z_1)$ in $R^3$ that will be in the plane and giving a vector $U=(a,b,c)\neq O$ that is perpendicular to the plane.

\vskip 1mm
When we say $U$ is perpendicular to the green plane we mean that the line segment $OU$ is perpendicular to the plane.

\vskip 1mm
Given any point $P=(x,y,z)$, the condition for $P$ to be in the green plane is that $AP\perp OU$ and this says that

$$\vec{AP}\cdot\vec{OU}=0$$

and this says that

$$(x-x_1,y-y_1,z-z_1)\cdot(a,b,c)=0$$

and we get

$$a(x-x_1)+b(y-y_1)+c(z-z_1)=0$$

\filbreak
\vskip 1cm
{\bf Parametric Equations of a Line}

\vskip 1mm
We can specify a line in $R^3$ by giving a point $A=(x_1,y_1,z_1)$ in the line and giving a vector $U=(a,b,c)\neq O$ that is parallel to the line.

\vskip 1mm
The condition for a point $P-(x,y,z)$ to be in the given line is that $AP$ is parallel to $OU$.

This says that for some number $t$ we have

$$\vec{AP}=t\vec{OU}$$

and this says that

$$(x-x_1,y-y_1,z-z_1)=t(abc)$$

Maybe we might like to write this equation in this form

$$\eqalign{x-x_1&=ta\cr
		y-y_1&=tb\cr
		z-z_1&=tc\cr}$$

and this says that

$$\eqalign{x&=x_1+at\cr
		y&=y_1+bt\cr
		z&=z_1+ct\cr}$$

\filbreak
\vfill\eject
\bye
