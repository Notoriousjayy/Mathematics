\nopagenumbers
{\bf Theorem: Properties of Functions Defined by Power Series}
\vskip 6pt
If the function $$f(x)=\sum_{n=0}^\infty a_n(x-c)^n= a_0+a_1(x-c)+a_2(x-c)^2+a_3(x-c)^3+\ldots$$ has a radius of convergence of $R>0$, then, on the interval $$(c-R,c+R)$$ $f$ is differentiable (and therefore continuous). Moreover, the derivative and antiderivative of $f$ are as follows.
$$1.\;\;\; f'(x)=\sum_{n=1}^\infty na_n(x-1)^{n-1}=a_1+2a_2(x-c)+3a_3(x-c)^2+\ldots$$
$$2.\;\;\; f(x)dx= C+\sum_{n=0}^\infty a_n{(x-c)^{n+1}\over n+1}=C+a_0(x-c)+a_1{)x-c)^2\over 2}+a_2{(x-c)^3\over 3}+\ldots$$
The {\it radius of convergence} of the series obtained by differentiating or integrating a power series is the same as that of the original power series. The {\it interval of convergence}, however, may differ as a result of the behavior at the endpoints.
\vfill\eject
