\nopagenumbers
{\bf Theorem: The Form of a Convergent Power Series}
\vskip 6pt
If $f$ is represented by a power series $f(x)=\sum a_n(x-c)^n$ for all $x$ in an open interval $I$ containing $c$. then $$a_n={f^{(n)}(c)\over n!}$$ and $$ f(x)=f(c)+f'(c)(x-c)+{f''(c)\over 2!}(x-c)^2+\ldots+{f^{(n)}(c)\over n!}(x-c)^n+\ldots$$
\vskip 10pt
{\bf Proof:}
\vskip 6pt
Consider a power series$\sum a_n(x-c)^n$ that has a radius of convergence $R$. Then, by Theorem 9.21, you know that the $n^{th}$ derivative of $f$ exists for $|x-c|<R$, and by successive differentiation you obtain the following.
$$f^{(0)}(x)=a_0+a_1(x-c)+a_2(x-c)^2+a_3(x-c)^3+a_4(x-c)^4+\ldots$$
$$f^{(1)}(x)=a_1+2a_2(x-c)+3a_(x-c)^2+4a_4(x-c)^3+\ldots$$
$$f^{(2)}(x)=2a_2+3!a_3(x-c)+4\bullet 3a_4(x-c)^2+\ldots$$
$$f^{(3)}(x)=3!a_3+4!a_4(x-c)+\ldots$$
$$\vdots$$
$$f^{(n)}(x)=n!a_n+(n+1)!a_{n+1}(x-c)+\ldots$$
Evaluating each of these derivatives at $x=c$ yields
$$f^{(0)}(c)=0!a_0$$
$$f^{(1)}(c)=1!a_1$$
$$f^{(2)}(c)=2!a_2$$
$$f^{(3)}(c)=3!a_3$$
and, in general, $f^{(n)}(c)=n!a_n$, By solving for $a_n$, you find the coefficients of the power series representation of $f(x)$ are $$a_n={f^{(n)}(c)\over n!}$$

\vfill\eject
