\nopagenumbers
{\bf Number of Solutions of a Linear System}
\vskip 1mm
\hrule

\vskip 6pt
For a system of linear equations, precisely one of the statements below is true.

\vskip 6pt
(1) The system has exactly one solution.

\vskip 6pt
(2) The system has infinitely many solutions.

\vskip 6pt
(3) The system has no solution

\vskip 10pt
{\bf Proof}

\vskip 6pt
Represent the system by the matrix operation $Ax=b$. If the system has exactly one solution or no solution, then there is nothing to prove. So, assume that the system has at least two distinct solutions $x_1$ and $x_2$. If you show that this assumption implies that the system has infinitely man solutions, then the proof is complete. When $x_1$ and $x_2$ are solutions, you have $Ax_1=Ax_2=b$ and $A(x_1-x_2)=0$. This implies that the (nonzero) column matrix $x_h=x_1-x_2$ is a solution of the homogeneous system of linear equations $Ax=0$. So, for any scalar $c$,

$$A(x_1+cx_h)=Ax_1+A(cx_h)=b+c(Ax_h)=b+c0=b$$

Then $x_1+cx_h$ is a solution of $Ax=b$ for any scalar at $c$. There are infineitely many possible values of $c$ and each va;ue produces a different solution, so the system has infinitely many solutions.

\vfill\eject
\bye
