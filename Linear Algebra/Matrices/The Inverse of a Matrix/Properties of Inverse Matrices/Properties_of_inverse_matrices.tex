\nopagenumbers
{\bf Theorem: Properties of Inverse Matrices}
\vskip 1mm
\hrule

\vskip 6pt
If $A$ is an invertible matric, $k$ is a positive integer, and $c$ is a nonzero scalar, then $A^{-1},A^k,cA$, and $A^T$ are invertible and the statements below are true.

\vskip 6pt
1. $(A^{-1})^{-1}=A$

\vskip 6pt
2. $(A^k)^{-1}=\underbrace{A^{-1}A^{-1}\cdots A^{-1}}_{k\rm\;factors}=(A^{-1})^k $

\vskip 6pt
3. $(cA)^{-1}={1\over c}A^{-1}$

\vskip 6pt
4. $(A^T)^{-1}=(A^{-1})^T$

\vskip 10pt
{\bf Proof}

\vskip 6pt
The key to the proofs of Properties 1,3, and 4 is the fact that the inverse of a matrix is unique. That is, if $BC=CB=I$, then $C$ is the inverse of $B$.

\vskip 1mm
Property 1 states that the inverse of $A^{-1}$ is $A$ itself. To prove this, observe that $A^{-1}A=AA^{-1}=I$, which means that $A$ is the inverse of $A^{-1}$. Thus $A=(A^{-1})^{-1}$.

\vskip 1mm
Similarily, Property 3 states that ${1\over c}A^{-1}$ is the inverse of $cA$, $c\neq 0$. To prove this, use the properties of scalar multiplication.

$$\eqalign{(cA)\Biggl({1\over c}A^{-1}\Biggr)&=\Biggl(c{1\over c}AA^{-1}\Biggr)=(1)I=I\cr
	\Biggl({1\over c}A^{-1}\Biggr)(cA)&=\Biggl(c{1\over c}AA^{-1}\Biggr)A^{-1}A=(1)I=I\cr}$$

So ${1\over c}A^{-1}$ is the inverse of $(cA)$, which implies that $(cA)^{-1}={1\over c}A^{-1}$.

\vfill\eject
\bye
