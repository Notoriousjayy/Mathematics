\nopagenumbers
{\bf Theorem: Covering a Board with Trominoes}
\vskip 1mm
\hrule

\vskip 6pt
For any integer $n\geq 1$, if one square is removed from a $2^n\times 2^n$ checkerboard, the remaining squares can be completely covered by L-shaped trominoes.

\vskip 10pt
{\bf Proof (by mathematical induction)}

\vskip 6pt
Let the property $P(n)$ be the sentence

\vskip 1mm
\centerline{If any square is removed from a $2^n\times 2^n$ checkerboard,}

\vskip 1mm
\centerline{then the remaining squares can be completely covered.}

\vskip 1mm
\centerline{by L-shaped trominoes.}

\vskip 6pt
{\bf Show that $P(1)$ is true};

\vskip 1mm
A $2^1\times 2^1$ chekerboard just consists of any four squares. If one square is removed, the remaining squares form an L, which can be covered by a single L-shaped tromino. Hence $P(1)$ is true.

\vskip 1pc
{\bf Show that for all integers $k\geq 1$, if $P(k)$is true then $P(k+1)$ is also true}:

\vskip 1mm
[Suppose that $P(k)$ is true for a particular but arbitrarily chosen integer $k\geq 3$. That is:] Let $k$ be any integer such that $k\geq 1$, and suppose that

\vskip 1mm
\centerline{If any square is removed from a $2^n\times 2^n$ checkerboard,}

\vskip 2mm
\centerline{then the remaining squares can be completely covered. $\quad\quad\quad\quad\quad\quad\quad\longleftarrow P(k)$}

\vskip 1mm
\centerline{by L-shaped trominoes.}

\vskip 3mm
$P(k)$ is the inductive hypothesis.

\vskip 1mm
[We must show that $P(k+1)$ is true, That is:] We must show that

\vskip 2mm
\centerline{If any square is removed from a $2^{k+1}\times 2^{k+1}$ checkerboard,}

\vskip 1mm
\centerline{then the remaining squares can be completely covered. $\quad\quad\quad\quad\quad\quad\quad\longleftarrow P(k+1)$}

\vskip 1mm
\centerline{by L-shaped trominoes.}

\vskip 3mm
Consider a $2^{k+1}\times 2^{k+1}$ checkerboard with one square removed. Divide it into four equal quadrants. Each will consist of a $2^{k+1}\times 2^{k+1}$ checkerboard. In one of the quadrants, one square will have been removed, and so, by inductive hypothesis, all the remaining squares in this quadrant can be completely covered by L-shaped trominoes. The other three quadrants meet at the center of the checkerboard, and the center of the checkerboard serves as a corner of a square from each of those quadrants. An L-shaped tromino can, therefore, be placed on those three central squares. By inductive hypoyhesis, the remaining squares in each of the three quadrants can be ompletely covered by L-shaped trominoes. Thus every square in the $2^{k+1}\times 2^{k+1}$ checkerboard except the one that was removed can be completely covered by L-shaped trominoes.



\vfill\eject
