\nopagenumbers
{\bf Example 6:} Show that $\phi(x)=\sin(x)-\cos(x)$ is a solution to the initial value problem

$${d^2y\over dx^2}+y=0;\quad y(0)=-1,\quad {dy\over dx}(0)=1$$


\vskip 10pt
{\bf Solution}

\vskip 6pt
Observe that $\phi(x)=\sin(x)-\cos(x),{d\phi\over dx}=\cos(x)+\sin(x)$, and ${d^2\phi\over dx^2}=-\sin(x)+\cos(x)$ are all defined on $(-\infty,\infty)$. Substituting into the differential equation gives

$$\biggl(-\sin(x)+\cos(x)\biggr)+\biggl(\sin(x)-\cos(x)\biggr)=0$$

which holds for all $x\in (-\infty,\infty)$. Hence, $\phi(x)$ is a solution to the differential equation in ${d^2y\over dx^2}+y=0;\quad y(0)=-1,\quad {dy\over dx}(0)=1$ on $(-\infty,\infty)$. When we check the initial conditions, we find

$$\eqalign{\phi(0)&=\sin(0)-\cos(0)=-1\cr
	{d\phi\over dx}(0)&=\cos(0)+\sin(0)=1}$$

Which meets the requirements of ${d^2y\over dx^2}+y=0;\quad y(0)=-1,\quad {dy\over dx}(0)=1$. Therefore, $\phi(x)$ is a solution to the given initial value problem.



\vfill\eject
\bye
