\nopagenumbers
{\bf Example 1:} Show that $\phi(x)=x^2-x^{-1}$ is an explicit solution to the linear equation

$${d^2y\over dx^2}-{2\over x^2}y=0$$

but $\psi(x)=x^3$ is not.

\vskip 10pt
{\bf Solution}

\vskip 6pt
The function $\phi(x)=x^2-x^{-1},\phi'(x)=2x+x^{-2},and \phi''(x)=2-2x^{-3}$ are defined for all $x\neq 0$. Substitution of $\phi(x)$ for $y$ in the equation gives

$$(2-2x^{-3})-{2\over x^2}(x^2-x^{-1})=(2-2x^{-3})-(2-2x^{-3})=0$$

Since this is valid for any $x\neq 0$, the function $\phi(x)=x^2-x^{-1}$ is an explicit solution to the equation on $(-\infty,0)$ and also on $(0,\infty)$.

\vskip 1mm
For $\psi(x)=x^3$ we have $\phi'(x)=3x^2,\psi''(x)=6x$ and substitution into the equation gives

$$6x-{2\over x^2}x^3=4x=0$$

which is valid only at the point $x=0$ and not on an interval. Hence $\psi(x)$ is not a solution.

\vfill\eject
\bye
