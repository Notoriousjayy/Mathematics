\nopagenumbers
{\bf Example 8:} For the initial value problem

$$3{dy\over dx}=x^2-xy^3,\quad y(1)=6$$

does the Existence and Uniqueness of Solution Theorem imply the existence of a unique solution?


\vskip 10pt
{\bf Solution}

\vskip 6pt
Dividing by $3$ to conform to the statement of the theorem, we identify $f(x,y)$ as ${(x^2-xy^3)\over 3}$ and ${\partial f\over \partial y}$ as $-xy^2$. Both of these functions are continuous in any rectangle containing the point $(1,6)$, so the hypotheses of the Existence and Uniquenes of Solution Theorem are satisfied. It then follows from the theorem that the initial value problem $3{dy\over dx}=x^2-xy^3,\quad y(1)=6$ has a unique solution in an interval about $x=1$ of the form $(1-\delta,1+\delta)$, where $\delta$ is some positive number.

\vfill\eject
\bye
