\nopagenumbers
{\bf Example 2:} Solve the initial value problem

$${dy\over dx}={y-1\over x+3},\quad y(-1)=0$$

\vskip 10pt
{\bf Solution}

\vskip 6pt
Separating the variables and integrating gives

$$\eqalign{{dy\over y-1}&={dx\over x+3}\cr
	\int {dy\over y-1}&=\int {dx\over x+3}\cr
	\ln|y-1|&=\ln|x+3|+C\cr}$$

At this point, we can either solve for $y$ explicitly (retaining the constant $C$) or use the initial condition to determine $C$ and then solve explicitly for $y$. Let's try the first approach.

\vskip 1mm
Exponentiating $\ln|y-1|=\ln|x+3|+C$, we have

$$\eqalign{e^{\ln|y-1|}&=e^{\ln|x+3|+C}=e^Ce^{\ln|x+3|}\cr
	|y-1|&=e^C|x+3|=C_1|x+3|\cr}$$

where $C_1:=e^C$. Now depending on the values of $y$, we have $|y-1|=\pm(y-1)$; and similarly, $|x+3|=\pm(x+3)$. Thus $|y-1|=e^C|x+3|=C_1|x+3|$ can be written as

$$y-1=\pm C_1(x+3)\quad\hbox{or}\quad y=1\pm C_1(x+3)$$

where the choice of sign depends on the values of $x$ and $y$. because $C_1$ is a {it positive} constant (recall that $C_1=e^C>0$), we can replace $\pm C_1$ by $C$, where $C$ now represents an {\it arbitrary} nonzero constant. We then obtain

$$y=1+C(x+3)$$

Finally, we determine $C$ such that the initial condition $y(-1)=0$ is satisfied. Putting $x=-1$ and $y=0$ in equation $y=1+C(x+3)$ gives

$$0=1+C(-1+3)=1+2C$$

and so $C=-{1\over 2}$. Thus the solution to the initial value problem is

$$y=1-{1\over 2}(x+3)=-{1\over 2}(x+1)$$

\vskip 190bp
{\bf Alternative Approach}

\vskip 6pt
The second appraoch is to first set $x=-1$ and $y=0$ in $\ln|y-1|=\ln|x+3|+C$ and solve for $C$. In this case, we obtain

$$\eqalign{\ln|0-1|&=\ln|-1+3|+C\cr
		0=\ln(1)&=\ln(2)+C\cr}$$

and so $C=-\ln(2)$. Thus from $\ln|y-1|=\ln|x+3|+C$, the solution is given implicitly by

$$\ln(1-y)=\ln(x+3)-\ln(2)$$

Here we have replaced $|y-1|$ by $1-y$ and $|x+3|$ by $x+3$, since we are interested in $x$ and $y$ near the initial values $x=-1$, $y=0$ (for such values, $y-1<0$ and $x+3>0$). Solving for $y$, we find

$$\eqalign{\ln(1-y)&=\ln(x+3)-\ln(2)=\ln\biggl({x+3\over 2}\biggr)\cr
	1-y&={x+3 \over 2}\cr
	y&=1-{1\over 2}(x+3)=-{1\over 2}(x+1)}$$

which agrees with the solution $y=1-{1\over 2}(x+3)=-{1\over 2}(x+1)$ found by the first method.

\vfill\eject
\bye
