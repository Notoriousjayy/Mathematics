\nopagenumbers
{\bf Example 2}: Solve

$${dy\over dx}=G(ax+by)$$

then the substitution

$$z=ax+by$$

transforms the equation into a separable one.

\vskip 6pt
{\bf Solution}

\vskip 6pt
The right-hand side can be expressed as a function of $x-y$, that is,

$$y-x-1+(x-y+2)^{-1}=-(x-y)-y+\bigl[(x-y)+2\bigr]^{-1}$$

so let $z=x-y$. To solve for ${dy\over dx}$, we differentiate $z=x-y$ with respect to $x$ to obtain ${dz\over dx}=1-{dy\over dx}$, and so ${dy\over dx}=1-{dz\over dx}$. Substituting into ${dy\over dx}=G(ax+by)$ yields

$$1-{dz\over dx}=-z-1+(z+2)^{-1}$$

or

$${dz\over dx}=(z+2)-(z+2)^{-1}$$

Solving this separable equation, we obtain

$$\eqalign{\int{z+2\over (z+2)^2-1}dz&=\int dx\cr
	{1\over 2}\ln\bigl|(z+2)^2-1\bigr|&=x+C_1}$$

from which it follows that

$$(z+2)^2=Ce^{2x}+1$$

Finally, replacing $z$ by $x-y$ yields

$$(x-y+2)^2=Ce^{2x}+1$$

as an implicit solution to ${dy\over dx}=G(ax+by)$


\vfill\eject
\bye
