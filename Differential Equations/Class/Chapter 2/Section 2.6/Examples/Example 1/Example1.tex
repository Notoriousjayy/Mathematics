\nopagenumbers
{\bf Example 1}: Solve

$$(xy+y^2+x^2)dx-x^2dy=0$$

\vskip 6pt
A check will show that $(xy+y^2+x^2)dx-x^2dy=0$ is not separable, exact or linear. If we express $(xy+y^2+x^2)dx-x^2dy=0$ in the derivative form

$${dy\over dx}={xy+y^2+x^2\over x^2}={y\over x}+\Biggl({y\over x}\Biggr)^2+1$$

then we see that the right-hand side of ${dy\over dx}={xy+y^2+x^2\over x^2}={y\over x}+\Bigl({y\over x}\Bigr)^2+1$ is a function of just $\Bigl({y\over x}\Bigr)$. Thus, $(xy+y^2+x^2)dx-x^2dy=0$ is homogeneous.

\vskip 1mm
Now let $v={y\over x}$  and recall that ${dy\over dx}=v+x({dv\over dx})$. With these substitutions, ${dy\over dx}={xy+y^2+x^2\over x^2}={y\over x}+\Bigl({y\over x}\Bigr)^2+1$ becomes

$$v+x{dv\over dx}=v+v^2+1$$

The above equation is separable, and, on separating the variables and integrating, we obtain

$$\eqalign{\int{1\over v^2+1}dv&=\int{1\over x}dx\cr
	\arctan(x)&=\ln|x|+C}$$

Hence,

$$v=\tan(\ln|x|+C)$$

Finally, we substitute ${y\over x}$ for $v$ and solve for $y$ to get

$$y=x\tan(\ln|x|+C)$$

\vfill\eject
\bye
