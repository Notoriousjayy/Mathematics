\nopagenumbers
{\bf Example 3}: Solve

$$(1+e^xy+xe^xy)dx+(xe^x+2)dy=0$$

\vskip 10pt
{\bf Solution}

\vskip 6pt
Here $M=1+e^xy+xe^xy$ and $N=xe^x+2$. Because

$${\partial M\over\partial y}=e^x+xe^xy={\partial N\over\partial x}$$

$(1+e^xy+xe^xy)dx+(xe^x+2)dy=0$ is exact. If we now integrate $N(x,y)$ with respect to $y$, we obtain

$$F(x,y)=\int(xe^x+2)dy+h(x)=xe^xy+2y+h(x)$$

When we take the partial derivative with respect to $x$ and substitute for $M$, we get

$$\eqalign{{\partial F\over\partial x}(x,y)&=M(x,y)\cr
		e^xy+xe^xy+h'(x)&=1+e^xy+xe^xy\cr}$$

Thus, $h'(x)=1$, so we take $h(x)=x$. Hence, $F(x,y)=xe^xy+2y+x$, and the solution to $(1+e^xy+xe^xy)dx+(xe^x+2)dy=0$ is given implicitly by $xe^xy+2y+x=C$. In this case we can solve explicitly for $y$ to obtain $y={(C-x)\over 2+xe^x}$

\vfill\eject
\bye
