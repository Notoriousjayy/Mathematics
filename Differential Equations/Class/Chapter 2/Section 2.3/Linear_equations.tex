\nopagenumbers
{\bf Linear Equations}
\vskip 1mm
\hrule

\vskip 6pt
{\bf Method for Solving Linear Equations}:

\vskip 6pt
(a) Write the equation in the standard form

$${dy\over dx}P(x)y=Q(x)$$

(b) Calculate the integrating factor $\mu(x)$ by the formula

$$\mu(x)=\hbox{exp}\biggl\lbrack\int P(x)dx\biggr\rbrack$$

(c) Multiply the equation in standard form by $\mu(x)$ and, recalling that the left-hand side is just ${d\over dx}\lbrack\mu(x)y\rbrack$, obtain

$$\eqalign{\underbrace{\mu(x){dy\over dx}+P(x)\mu(x)y}&=\mu(x)Q(x)\cr
			{d\over dx}\lbrace\mu(x)y\rbrace&=\mu(x)Q(x)\cr}$$

(d) Integrate the last equation and solve for $y$ by dividing by $\mu(x)$ to obtain

$$y(x)={1\over\mu(x)}\biggl\lbrack\int\mu(x)Q(x)dx+C\biggr\rbrack$$

\vskip 6pt
{\bf Existence and Uniquenes of Solution}

\vskip 6pt
Suppose $P(x)$ and $Q(x)$ are continuous on an interval $(a,b)$ that contains the point $x_0$. Then for any choice of initial value $y_0$, there exists a unique solution $y(x)$ on $(a,b)$ to the initial value problem

$${dy\over dx}+P(x)y=Q(x), \quad y(x_0)=y_0$$

In fact the solution is given by $y(x)={1\over\mu(x)}\bigl\lbrack\int\mu(x)Q(x)dx+C\bigr\rbrack$ for a suitable value of $C$.

\vfill\eject
\bye
