\nopagenumbers
{\bf Theorem 5.11}
\vskip 6pt
For every graph $G$ $$ \kappa (G) \leq \lambda (G) \leq \delta (G)$$
\vskip 10pt
{\bf Proof:}
\vskip 6pt
If $G$ is disconnected or trivial, then $\kappa (G)= \lambda (G)=0$ and the inequalities hold; while ig $G=K_n$ for some integer $n \geq 2$, then $\kappa (G)= \lambda (G) = \delta (G=n-1$. Thus we may assume that $G$ is a connected graph of order $n \geq 3$ that is not complete. Hence $\delta (G)\leq n-2$.
\vskip 1mm
First, we show that $\lambda (G) \leq \delta (G)$. Let $v$ be a vertex of $G$ with $deg(v)= \delta (G)$. Since the set of the $\delta (G)$ edges incident with $v$ in $G$ is an edge-cut, it follows that $$\lambda (G) \leq \delta (G) \leq n-2$$
It remains to show that $\kappa (G) \leq \lambda (G)$. Let $X$ be a minimum edge-cut of $G$. Then $|X|= \lambda (G) \geq n-2$ Necessarily, $G-X$ contains exactly two components $G_1$ and $G_2$. Suppose that the order of $G_1$ is $k$. Thus the order of $G$ is $n-k$, where $k \geq 1$ and $n-k \geq 1$. Consequently, every edge in $X$ joins a vertex of $G_1$ and a vertex of $G_2$. We consider two cases.
\vskip 1mm
{\bf Case 1.} {\it Every vertex of $G_1$ is adjacent in $G$ to every vertex of $G_2$}. Thus $|X|=k(n-k)$. Since $(k-1)(n-k-1) \geq 0$, it follows that $$(k-1)(n-k-1)=k(n-k)-n+1 \geq 0$$ and so $\lambda (G)= |X|=k(n-k) \geq n-1$. However, $\lambda (G) \leq n-2$; so this case cannot occur.
\vskip 1mm
{\bf Case 2.} {\it There exists vertices in $u$ in $G_1$ and $v$ in $G_2$ such that $u$ and $v$ are not adjacent in $G$}. We now define a set $U$ of vertices of $G$. For each $e \in X$, we select a vertex for $U$ in the following way. If $u$ is incident with $e$, then choose the other vertex in $G_2$ that is incident with $e$ as an element of $U$; otherwise, select the vertex that is incident with $e$ and belongs to $G_1$ as an element of $U$. it follows that $G-U$ is disconnected and so $U$ is a vertex-cut. Hence $$ \kappa (G) \leq |U| \leq |X|= \lambda (G)$$ as desired.

\vfill\eject
