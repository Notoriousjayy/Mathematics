\nopagenumbers
{\bf Theorem 5.16 (Menger's Theorem)}
\vskip 6pt
Let $u$ and $v$ be non-adjacent vertices in a graph $G$. The {\it minimum number} of vertices in a $u$---$v$ separating set equals the maximum number of internally disjoint $u$---$v$ paths in $G$.
\vskip 10pt
{\bf Proof:}
\vskip 6pt
We proceed by induction on the size of the graph. Certainly, the result is true vacuously for all empty graphs. Assume that the result is true for all graphs of size {\it less} than $m$, where $m$ is a positive integer, and let $G$ be a graph of size $m$. Let $u$ and $v$ be two non-adjacent vertices of $G$. Suppose that there are $k$ vertices in a minimum $u$---$v$ spanning set. Certainly, $G$ can contain no more than $k$ internally disjoint $u$---$v$ paths. We show, in fact, that $G$ contains $k$ internally disjoint $u$---$v$ paths. Since the result is true for $k=0$ and $k=1$, we may assume that $k \geq 2$. We consider three cases.
\vskip 1mm
{\bf Case 1.} {\it There exists a minimum $u$---$v$ separating set $U$ in $G$ containing a vertex $x$ that is adjacent to both $u$ and $v$}. Then the size of the subgraph $G-x$ is less than $m$ and $U - \lbrace x \rbrace$ is a minimum $u$---$v$ separating set in $G-x$ consisting of $k-1$ vertices. By the induction hypothesis, there are $k-1$ internally disjoint $u$---$v$ paths in $G-x$. These paths together with the path $(u,x,v)$ constitute $k$ internally disjoint $u$---$v$ paths in $G$.
\vskip 1mm
{\bf Case 2.} {\it There exists a minimum $u$---$v$ separating set $W$ in $G$ containing a vertex in $W$ that is not adjacent to $u$ and a vertex in $W$ that is not adjacent to $v$}. Let $W= \lbrace w_1,w_2, \ldots , w_k \rbrace$. Let $G_u$ be the subgraph of $G$ consisting of all $u$---$w_i$ paths in $G$, where only $w_i \in W$ for each $i(1 \leq i \leq k)$ and let $G'_u$ be the graph obtained from $G_u$ by adding a new vertex $v'$ and joining $v'$ to each vertex $w_i$ for $1 \leq i \leq k$. Let $G_v$ and $G'_v$ be defined similarly, where $G'_v$ is obtained from $G_v$ by adding the new vertex $u'$.
\vskip 1mm
Since $W$ contains a vertex that is not adjacent to $u$ and a vertex that is not adjacent to $v$, the size of each of the graphs $G'_u$ and $G'_v$ is less than $m$. Since $W$ is a minimum $u$---$v$ separating set in $G'_u$, it follows by the induction hypothesis that $G'_u$ contains $k$ internally disjoint $u$---$v'$ paths, each consisting of a $u$---$w_i$ path $P_i$ followed by the edge $w_iv'$. Similarly, there are $k$ internally disjoint $u'$---$v$ paths in $G'_v$, each consisting og a $w_i$---$v$ path $Q_i$ preceeded by the edge $u'w_i$. Since $W$ is a $u$---$v$ separating set in $G$, the two graphs $G_u$ and $G_v$ have only the vertices of $W$ in common. Therefore, the $k$ paths obtained by following $P_i$ by $Q_i$ for each $i(1 \leq i \leq k)$ are internally disjoint $u$---$v$ paths in $G$.
\vskip 1mm
{\bf Case 3.} {\it For each minimum $u$---$v$ separating set $S$ in $G$, either every vertex of $S$ is adjacent to $u$ and not adjacent to $v$ or every vertex of $S$ is adjacent to $v$ and not adjacent to $u$}. Let $P=(u,x,y, \ldots ,v)$ be a $u$---$v$ geodesic in $G$ and let $e=xy$. Consider the subgraph $G-e$ in $G$. Certainly, every minimum $u$---$v$ separating set in $G-e$ contains $k-1$ vertices. We claim, in fact, that a minimum $u$---$v$ separating set in $G-e$ contains $k$ vertices, fro assume to the contrary, that $G-e$ contains a minimum $u$---$v$ separating set $Z= \lbrace z_1,z_2, \ldots ,z_{k-1})$. Then $Z \cup \lbrace x \rbrace$ is a minimum $u$---$v$ separating set in $G$. Since $x$ is adjacent to $v$. On the other hand, $Z \cup \lbrace y \rbrace$ is a minimum $u$---$v$ separating set in $G$ and each $z_i(1 \leq i \leq k-1)$ is adjacent to $u$. This implies that $y$ is also adjacent to $u$, which contradicts the fact that $P$ is a $u$---$v$ geodesic. Therefore, as claimed, $k$ is the minimum number of vertices in a $u$---$v$ separating set in $G-e$. By the induction hypothesis, there are $k$ internally disjoint $u$---$v$ separating set in $G-e$. Hence there are $k$ internally disjoint $u$---$v$ paths in $G$ as well.

\vfill\eject
