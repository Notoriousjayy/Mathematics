\nopagenumbers
{\bf Theorem 5.5}
\vskip 6pt
Let $G$ be a non-trivial connected graph and let $u \in V(G)$. If $v$ is a vertex that is farthest from $u$ in $G$, then $v$ is not a cut-vertex of $G$.
\vskip 10pt
{\bf Proof:}
\vskip 6pt
Assume, to the contrary, that $v$ is a cut-vertex of $G$. Let $w$ be a vertex belonging to a component of $G-v$ that does not contain $u$. Since every $u$---$w$ path contains $v$, it follows that $d(u,w) > d(u,v)$, which is a contradiction.
\vskip 10pt
{\bf Corollary 5.6}
\vskip 6pt
Every non-trivial connected graph contains at least two vertices that are not cut-vertices.
\vskip 10pt
{\bf Proof:}
\vskip 6pt
Let $u$ and $v$ be vertices of a non-trivial connected graph $G$ such that $d(u,v)=diam(G)$. Since each of $u$ and $v$ is farthest from the other, it follows by Theorem 5.5 that both $u$ and $v$ are not cut-vertices of $G$.



\vfill\eject
