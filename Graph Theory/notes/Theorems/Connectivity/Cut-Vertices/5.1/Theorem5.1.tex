\nopagenumbers
{\bf Theorem 5.1}
\vskip 6pt
Let $v$ be a vertex incident with a bridge in a connected graph $G$. Then $v$ is a cut-vertex of $G$ {\it if and only if} $deg(v) \geq 2$.
\vskip 10pt
{\bf Proof:}
\vskip 6pt
Suppose that $uv$ is a bridge of $G$. Then $deg(v) \geq 1$. Assume that $deg(v)=1$. Since $v$ is an end-vertex of $G$, the graph $G-v$ is connected and so $v$ is not a cut-vertex of $G$.
\vskip 1mm
For the converse, assume that $deg(v) \geq 2$. Then there is a vertex $w$ different from $u$ that is adjacent to $v$. Assume, to the contrary, that $v$ is not a cut-vertex. Thus $G-v$ is connected and so there is a $u$---$v$ path $P$ in $G$---$v$. However then, $P$ together with $v$ and the two edges $uv$ and $vw$ form a cycle containing the bridge $uv$. This contradicts Theorem 4.1.

\vskip 10pt
{\bf Corollary 5.2}
\vskip 6pt
Let $G$ be a connected graph or order $3$ or more. If $G$ {\it contains a bridge, then $G$ contains a cut-vertex}.
\vskip 10pt
If $v$ is a cut-vertex in a connected graph $G$, then, of course $G-v$ contains two or more components. If $u$ and $w$ are vertices in distinct components of $G-v$, then $u$ and $w$ are not connected in $G-v$. On the other hand, $u$ and $w$ are necessarily connected in $G$.


\vfill\eject
